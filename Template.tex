% Copyright (c) 2019 Bochen Tan
% Public domain.
%本模板的宗旨是尽量绿色,不需要附加安装任何东西。
%按照教务部下发的WORD说明文档格式,下简称“说明”
%没有封面和评阅表,这两部分请直接在Cover&ReviewTable.doc中写再输出pdf拼到一起
%doc小改动:封面校徽和文字替换为了高清版本,“题目:”和中文题目对齐,中英文题目分在了表的两行
%doc小改动:插入了两个白页,使得连续打印的时候封面和表格都在奇数页
%正文部分改动:在每一页下方中央加了页码,因为说明中页眉不分奇偶页,所以页码就都在中央吧
%不含自动的参考文献,因为说明中参考文献格式不典型,请手动输入或自行写程序
%在Windows或Linux下渲染出字体更接近说明,Mac OS上字体不太一样
%有警告\headheight is too small,fancyhdr的上距离有点小,似乎问题不大

\documentclass[UTF8,openany,AutoFakeBold,AutoFakeSlant,cs4size]{ctexbook}
%openany 使一章可以从偶数页开始,因为说明中每一章并没有只能从奇数页开始,虽然这是常理
%AutoFakeBold 和 AutoFakeSlant 因为 CJK 里没有真正的加粗和倾斜,如果额外字体则效果更好
%cs4size 因为要求主题是小四号字

\usepackage[a4paper,left=3.18cm,right=3.18cm,top=2.54cm,bottom=2.54cm]{geometry}
%office中正常页边距



\usepackage{amsmath}
\usepackage{bm}
\usepackage{amsfonts}
\usepackage{cite}
\usepackage{enumerate}
\usepackage{fancyhdr}



\usepackage{multirow,booktabs,makecell}
\usepackage{graphicx}
\usepackage[font=small,labelsep=space]{caption} %五号,宋体/Time new roman
\renewcommand{\thetable}{\arabic{table}} %表格和图片编号不分章节,直接1,2,3 ...
\renewcommand{\thefigure}{\arabic{table}}



\usepackage{tocloft} %自定义目录,说明中的目录没有按照字面写的设置,这里按字面写的为准

\setlength\cftparskip{10pt} %目录项之间垂直距离,不确定,待改

\renewcommand\cftchapfont{\heiti\zihao{2}} %二号,黑体
\renewcommand\cftchapdotsep{\cftdotsep} %有点连到页码,点间距不确定,待改
\renewcommand\cftchappagefont{\bfseries\songti\zihao{2}}
%页码的文字用加粗宋体,因为\heiti下英文,数字和标点并没有被加粗,与WORD不同

%1. 第一级 三号宋体/Time new roman、加粗
\renewcommand\cftsecfont{\bfseries\songti\zihao{3}}
\renewcommand\cftsecpagefont{\bfseries\songti\zihao{3}}

%1.1 第二级 四号,宋体/Time new roman,加粗,页码一致
\renewcommand\cftsubsecfont{\bfseries\songti\zihao{4}}
\renewcommand\cftsubsecpagefont{\bfseries\songti\zihao{4}}

%1.1.1 第三级 小四,宋体/Time new roman加粗,页码一致
\renewcommand\cftsubsubsecfont{\bfseries\songti\zihao{-4}}
\renewcommand\cftsubsubsecpagefont{\bfseries\songti\zihao{-4}}



\usepackage{titlesec}%自定义章节标题
\CTEXsetup[format={\center\heiti\zihao{2}},beforeskip=0pt]{chapter}
%第一章  绪论(二号、黑体) beforeskip为上方垂直距离看起来还比说明偏大,待改

\setcounter{tocdepth}{3}
\setcounter{secnumdepth}{3}
%使目录中有三级标题,即subsubsection

\renewcommand\thesection{\arabic{section}} % 使得不显示章名,只显示节名
\titleformat{\section}
{\raggedright\zihao{3}\bfseries\songti}
{\thesection.\quad}
{0pt}
{}%1. 第一级(三号、宋体/Time new roman、加粗)

\titleformat{\subsection}
{\raggedright\bfseries\zihao{4}\songti}
{\thesubsection\quad}
{0pt}
{}%1.1 第二级(四号,宋体/Time new roman,加粗)

\titleformat{\subsubsection}
{\raggedright\bfseries\zihao{-4}\songti}
{\thesubsubsection\quad}
{0pt}
{}%1.1.1 第三级(小四,宋体/Time new roman,加粗)





\title{}
\author{}
\date{}
\begin{document}





\chapter*{版权声明}
\setcounter{page}{0}
% 本页不计页码
\thispagestyle{empty}
% 本页无页眉和页脚
任何收存和保管本论文各种版本的单位和个人,未经本论文作者同意,不得将本论文转借他人,亦不得随意复制、抄录、拍照或以任何方式传播。否则,引起有碍作者著作权之问题,将可能承担法律责任。
\clearpage

%版权声明后空白一页,使得摘要从奇数页开始。
\quad
\setcounter{page}{0}
% 本页不计页码
\thispagestyle{empty}
% 本页无页眉和页脚
\clearpage



\pagestyle{fancy}
\normalsize
\linespread{1.5}
%小四号,宋体/Time new roman,1.5倍行距
\chapter*{摘要}
Copyright (c) 2019 Bochen Tan

Public domain.

本模板的宗旨是尽量绿色,不需要附加安装任何东西。

按照教务部下发的WORD说明文档格式,下简称“说明”

没有封面和评阅表,这两部分请直接在Cover\&ReviewTable.doc中写再输出pdf拼到一起

doc小改动:封面校徽和文字替换为了高清版本,“题目:”和中文题目对齐,中英文题目分在了表的两行

doc小改动:插入了两个白页,使得连续打印的时候封面和表格都在奇数页

正文部分改动:在每一页下方中央加了页码,因为说明中页眉不分奇偶页,所以页码就都在中央吧

不含自动的参考文献,说明中参考文献格式不典型,请手动输入或自行写程序

在Windows或Linux下渲染出字体更接近说明,Mac OS上字体不太一样

有警告$\backslash$ headheight is too small,fancyhdr的上距离有点小,似乎问题不大

\bigskip
\noindent{\bfseries\songti 关键词: } 



\addcontentsline{toc}{chapter}{摘要} %手动加入目录
\fancypagestyle{plain} %因为latex默认每章第一页是plain所以需要重置一下plain和说明统一
{
	\fancyhf{} %清空
	
	\fancyhead[RE,RO]{摘要}
	%偶数页右页眉,奇数页右页眉均为“摘要”,及章名\leftmark

	\fancyhead[LE,LO]{北京大学本科生毕业论文}
	%偶数页左页眉,奇数页左页眉均为“北京大学本科生毕业论文”
	
	\fancyfoot[CO,CE]{~\thepage~}
	%偶数页和奇数页中页脚为页码,从对称考虑,因为每页在说明中都是一样的,不分奇偶
	
	\renewcommand{\headrulewidth}{0.7pt} %页眉线宽度,可调,不太清楚说明中是多少,待改
	
	\renewcommand{\footrulewidth}{0pt} %页脚线宽度为0,既没有
}

%默认的风格是fancy,设置于下,用于每章非第一页
\fancyhf{}
\fancyhead[RE,RO]{摘要}
\fancyhead[LE,LO]{北京大学本科生毕业论文}
\fancyfoot[CO,CE]{~\thepage~}
\renewcommand{\headrulewidth}{0.7pt}
\renewcommand{\footrulewidth}{0pt}
\clearpage






\small
\linespread{1.5}
%5号,Time new roman,1.5倍行距
\chapter*{\bfseries Abstract}

\bigskip
\noindent
{\bfseries Key Words: }


	
\addcontentsline{toc}{chapter}{\bfseries Abstract} %Abstract加粗
\fancypagestyle{plain}
{
	\fancyhf{}
	\fancyhead[RE,RO]{Abstract}
	\fancyhead[LE,LO]{北京大学本科生毕业论文}
	\fancyfoot[CO,CE]{~\thepage~}
	\renewcommand{\headrulewidth}{0.7pt}
	\renewcommand{\footrulewidth}{0pt}
}
\fancyhf{}
\fancyhead[RE,RO]{Abstract}
\fancyhead[LE,LO]{北京大学本科生毕业论文}
\fancyfoot[CO,CE]{~\thepage~}
\renewcommand{\headrulewidth}{0.7pt}
\renewcommand{\footrulewidth}{0pt}
\clearpage





\fancypagestyle{plain}
{
	\fancyhf{}
	\fancyhead[RE,RO]{全文目录}
	\fancyhead[LE,LO]{北京大学本科生毕业论文}
	\fancyfoot[CO,CE]{~\thepage~}
	\renewcommand{\headrulewidth}{0.7pt}
	\renewcommand{\footrulewidth}{0pt}
}
\fancyhf{}
\fancyhead[RE,RO]{全文目录}
\fancyhead[LE,LO]{北京大学本科生毕业论文}
\fancyfoot[CO,CE]{~\thepage~}
\renewcommand{\headrulewidth}{0.7pt}
\renewcommand{\footrulewidth}{0pt}
\renewcommand{\contentsname}{\centerline{全文目录}}
\tableofcontents
\addcontentsline{toc}{chapter}{全文目录}
\clearpage





\normalsize
\linespread{1.5}
%正文,小四号,中文宋体,英文Time new roman,1.5倍行距
\fancypagestyle{plain}
{
	\fancyhf{}
	\fancyhead[RE,RO]{\leftmark}
	\fancyhead[LE,LO]{北京大学本科生毕业论文}
	\fancyfoot[CO,CE]{~\thepage~}
	\renewcommand{\headrulewidth}{0.7pt}
	\renewcommand{\footrulewidth}{0pt}
}
\fancyhf{}
\fancyhead[RE,RO]{\leftmark}
\fancyhead[LE,LO]{北京大学本科生毕业论文}
\fancyfoot[CO,CE]{~\thepage~}
\renewcommand{\headrulewidth}{0.7pt}
\renewcommand{\footrulewidth}{0pt}



\chapter{章节名称}
\section{一级段落名称}
\subsection{二级段落名称}
\subsubsection{三级段落名称}
引用如\cite{PhysRev.47.777},引用表如表\ref{tab:input_output_r},引用图如图\ref{fig:sample}.

\begin{table}[h]
\small %内容,(五号,宋体/Time new roman)
\centering
\caption{不同频率下的输入和输出阻抗}
\label{tab:input_output_r}
\begin{tabular}{cccc} %表格使用三线表
\toprule %不确定说明中三条线的粗细,待改
频率(Hz) & 1 & 10k & 1M \\
\midrule
输入电阻($\Omega/^\circ$) & 339.719k/-87.84 & 5.6707k/-9.827 & 351.188/-72.377\\
输出电阻($\Omega/^\circ$) & 338.638k/-89.663 & 1.9866k/-1.1228 & 1.9189k/-14.801 \\
\bottomrule
\end{tabular}
\end{table}

\begin{figure}[h]
\centering
\includegraphics[width=12cm]{sample.jpg}
\caption{示例图片}
\label{fig:sample}
\end{figure}
\clearpage





\small
\linespread{1}
%正文,五号,中文宋体,英文Time new roman,1倍行距
\chapter*{参考文献}
\noindent
\begin{enumerate}[{[1]}]
\small
	\item 期刊
	\item 网络文档
\end{enumerate}

1.	期刊  作者. 论文名. 刊名, 出版年份, 卷号(期号): 起始-截止页

2.	专著  作者. 书名. 版本(第一版不写). 出版城市: 出版社, 出版年份: 起始-截止页

3.	论文集  论文作者. 论文题目//编者. 论文集名: 其他题名信息. 出版城市(或者会议城市): 出版者, 出版年: 引文起始-截止页码

4.	学位论文  作者. 学位论文题名. 城市: 论文保存单位, 年份

5.	网络文献  作者. 题名[文献类型标志/文献载体标志]. 出版地: 出版者, 出版年(更新日期)[引用日期]. 获取和访问路径

*注意: 作者姓前名后, 超过3名作者列前3名, 后加“, 等”; 英文姓名, 姓前名后, 姓首字母大写, 名缩写; 文献的项目要完整, 各项的顺序和标点要和格式要求一致; 未公开发表的论文、报告不列入正式文献, 如有必要可在正文当页下加注。英文文献格式同上。参考文献在正文中按出现顺序用[1], [2]......在右上角标注, 放在“参考文献”中时, 用[1], [2], ...顺序标注。



\addcontentsline{toc}{chapter}{参考文献}
\fancypagestyle{plain}
{
	\fancyhf{}
	\fancyhead[RE,RO]{参考文献}
	\fancyhead[LE,LO]{北京大学本科生毕业论文}
	\fancyfoot[CO,CE]{~\thepage~}
	\renewcommand{\headrulewidth}{0.7pt}
	\renewcommand{\footrulewidth}{0pt}
}
\fancyhf{}
\fancyhead[RE,RO]{参考文献}
\fancyhead[LE,LO]{北京大学本科生毕业论文}
\fancyfoot[CO,CE]{~\thepage~}
\renewcommand{\headrulewidth}{0.7pt}
\renewcommand{\footrulewidth}{0pt}






\bibliographystyle{plain}
\bibliography{ref}
这是参考“参考文献”,主要用来看引用的顺序,请手动些参考文献或自行写程序,最终编译请删除
\clearpage





\linespread{1}
\normalsize
%小四号,中文宋体,英文Time new roman,1倍行距
\chapter*{本科期间的主要工作和成果}

\noindent 本科期间参加的主要科研项目

\noindent 本研基金
\begin{enumerate}
	\item 基金名称. 基金类型. 指导老师. 基金支持年限
\end{enumerate}

\noindent 各种科研项目
\begin{enumerate}
	\item 项目名称. 项目类型
\end{enumerate}

格式下
期刊:
全部作者. 论文名. 期刊名, 出版年份, 卷号(期号): 起始-截止页
会议论文:
全部作者. 论文名. 会议名, 会议举办地, 会议举办时间, 起始-截止页
专利
全部专利申请人. 专利名称. 专利申请号. 专利申请日期. 国别



\addcontentsline{toc}{chapter}{本科期间的主要工作和成果}
\fancypagestyle{plain}
{
	\fancyhf{}
	\fancyhead[RE,RO]{本科期间的主要工作和成果}
	\fancyhead[LE,LO]{北京大学本科生毕业论文}
	\fancyfoot[CO,CE]{~\thepage~}
	\renewcommand{\headrulewidth}{0.7pt}
	\renewcommand{\footrulewidth}{0pt}
}
\fancyhf{}
\fancyhead[RE,RO]{本科期间的主要工作和成果}
\fancyhead[LE,LO]{北京大学本科生毕业论文}
\fancyfoot[CO,CE]{~\thepage~}
\renewcommand{\headrulewidth}{0.7pt}
\renewcommand{\footrulewidth}{0pt}
\clearpage





\linespread{1.5}
\normalsize
%正文,小四号,中文宋体,英文Time new roman,1.5倍行距
\chapter*{致谢}



\addcontentsline{toc}{chapter}{致谢}
\fancypagestyle{plain}
{
	\fancyhf{}
	\fancyhead[RE,RO]{致谢}
	\fancyhead[LE,LO]{北京大学本科生毕业论文}
	\fancyfoot[CO,CE]{~\thepage~}
	\renewcommand{\headrulewidth}{0.7pt}
	\renewcommand{\footrulewidth}{0pt}
}
\fancyhf{}
\fancyhead[RE,RO]{致谢}
\fancyhead[LE,LO]{北京大学本科生毕业论文}
\fancyfoot[CO,CE]{~\thepage~}
\renewcommand{\headrulewidth}{0.7pt}
\renewcommand{\footrulewidth}{0pt}





\end{document}