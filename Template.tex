% Copyright (c) 2019 Bochen Tan
% Public domain.
\documentclass[UTF8,openany,AutoFakeBold,AutoFakeSlant,cs4size]{ctexbook}
\usepackage{graphicx}
\usepackage[a4paper,left=3.18cm,right=3.18cm,top=2.54cm,bottom=2.54cm]{geometry}
\usepackage{amsmath}
\usepackage{multirow,booktabs,makecell}
\usepackage{graphicx}
\usepackage[font=small,labelsep=space]{caption}
\renewcommand{\thetable}{\arabic{table}}
\renewcommand{\thefigure}{\arabic{table}}
\usepackage{bm}
\usepackage{titlesec}
\usepackage{fancyhdr}
\usepackage{amsfonts}
\usepackage{cite}
\usepackage{enumerate}

\usepackage{tocloft}
\setlength\cftparskip{10pt}
\renewcommand\cftchapfont{\heiti\zihao{2}}
\renewcommand\cftchapdotsep{\cftdotsep}
\renewcommand\cftchappagefont{\bfseries\songti\zihao{2}}
\renewcommand\cftsecfont{\bfseries\songti\zihao{3}}
\renewcommand\cftsecpagefont{\bfseries\songti\zihao{3}}
\renewcommand\cftsubsecfont{\bfseries\songti\zihao{4}}
\renewcommand\cftsubsecpagefont{\bfseries\songti\zihao{4}}
\renewcommand\cftsubsubsecfont{\bfseries\songti\zihao{-4}}
\renewcommand\cftsubsubsecpagefont{\bfseries\songti\zihao{-4}}

\CTEXsetup[format={\center\heiti\zihao{2}},beforeskip=0pt]{chapter}

\renewcommand\thesection{\arabic{section}}
\setcounter{tocdepth}{3}
\setcounter{secnumdepth}{3}

\titleformat{\section}
{\raggedright\zihao{3}\bfseries\songti}
{\thesection.\quad}
{0pt}
{}

\titleformat{\subsection}
{\raggedright\bfseries\zihao{4}\songti}
{\thesubsection\quad}
{0pt}
{}

\titleformat{\subsubsection}
{\raggedright\bfseries\zihao{-4}\songti}
{\thesubsubsection\quad}
{0pt}
{}



\title{}
\author{}
\date{}
\begin{document}

\chapter*{版权声明}
\setcounter{page}{0}
\thispagestyle{empty}
任何收存和保管本论文各种版本的单位和个人,未经本论文作者同意,不得将本论文转借他人,亦不得随意复制、抄录、拍照或以任何方式传播。否则,引起有碍作者著作权之问题,将可能承担法律责任。
\clearpage
\quad
\setcounter{page}{0}
\thispagestyle{empty}
\clearpage



\pagestyle{fancy}
\chapter*{摘要}
\linespread{1.5}


\bigskip
\noindent{\bfseries\songti 关键词: } 
\addcontentsline{toc}{chapter}{摘要}
\fancypagestyle{plain}
{
	\fancyhf{}
	\fancyhead[RE,RO]{摘要}
	\fancyhead[LE,LO]{北京大学本科生毕业论文}
	\fancyfoot[CO,CE]{~\thepage~}
	\renewcommand{\headrulewidth}{0.7pt}
	\renewcommand{\footrulewidth}{0pt}
}
\fancyhf{}
\fancyhead[RE,RO]{摘要}
\fancyhead[LE,LO]{北京大学本科生毕业论文}
\fancyfoot[CO,CE]{~\thepage~}
\renewcommand{\headrulewidth}{0.7pt}
\renewcommand{\footrulewidth}{0pt}






\newpage
\small
\chapter*{\bfseries Abstract}
	\bigskip
	\noindent
	{\bfseries Key Words: }
\addcontentsline{toc}{chapter}{\bfseries Abstract}
\fancypagestyle{plain}
{
	\fancyhf{}
	\fancyhead[RE,RO]{Abstract}
	\fancyhead[LE,LO]{北京大学本科生毕业论文}
	\fancyfoot[CO,CE]{~\thepage~}
	\renewcommand{\headrulewidth}{0.7pt}
	\renewcommand{\footrulewidth}{0pt}
}
\fancyhf{}
\fancyhead[RE,RO]{Abstract}
\fancyhead[LE,LO]{北京大学本科生毕业论文}
\fancyfoot[CO,CE]{~\thepage~}
\renewcommand{\headrulewidth}{0.7pt}
\renewcommand{\footrulewidth}{0pt}






\newpage
\fancypagestyle{plain}
{
	\fancyhf{}
	\fancyhead[RE,RO]{全文目录}
	\fancyhead[LE,LO]{北京大学本科生毕业论文}
	\fancyfoot[CO,CE]{~\thepage~}
	\renewcommand{\headrulewidth}{0.7pt}
	\renewcommand{\footrulewidth}{0pt}
}
\fancyhf{}
\fancyhead[RE,RO]{全文目录}
\fancyhead[LE,LO]{北京大学本科生毕业论文}
\fancyfoot[CO,CE]{~\thepage~}
\renewcommand{\headrulewidth}{0.7pt}
\renewcommand{\footrulewidth}{0pt}
\renewcommand{\contentsname}{\centerline{全文目录}}
\tableofcontents
\addcontentsline{toc}{chapter}{全文目录}






\newpage
\normalsize
\fancypagestyle{plain}
{
	\fancyhf{}
	\fancyhead[RE,RO]{\leftmark}
	\fancyhead[LE,LO]{北京大学本科生毕业论文}
	\fancyfoot[CO,CE]{~\thepage~}
	\renewcommand{\headrulewidth}{0.7pt}
	\renewcommand{\footrulewidth}{0pt}
}
\fancyhf{}
\fancyhead[RE,RO]{\leftmark}
\fancyhead[LE,LO]{北京大学本科生毕业论文}
\fancyfoot[CO,CE]{~\thepage~}
\renewcommand{\headrulewidth}{0.7pt}
\renewcommand{\footrulewidth}{0pt}


\chapter{章节名称}
\section{一级段落名称}
\subsection{二级段落名称}
\subsubsection{三级段落名称}
引用如\cite{PhysRev.47.777},引用表如表\ref{tab:input_output_r},引用图如图\ref{fig:sample}.


\begin{table}[h]
\small
\centering
\caption{不同频率下的输入和输出阻抗}
\label{tab:input_output_r}
\begin{tabular}{cccc}
\toprule
频率(Hz) & 1 & 10k & 1M \\
\midrule
输入电阻($\Omega/^\circ$) & 339.719k/-87.84 & 5.6707k/-9.827 & 351.188/-72.377\\
输出电阻($\Omega/^\circ$) & 338.638k/-89.663 & 1.9866k/-1.1228 & 1.9189k/-14.801 \\
\bottomrule
\end{tabular}
\end{table}




\begin{figure}[h]
\centering
\includegraphics[width=12cm]{sample.jpg}
\caption{示例图片}
\label{fig:sample}
\end{figure}





\newpage
\small
\linespread{1}	
\chapter*{参考文献}
\noindent
\begin{enumerate}[{[1]}]
\small
	\item 期刊
	\item 网络文档
\end{enumerate}


说明:参考文献自己打出来,因为给的格式不常规


1.	期刊  作者. 论文名. 刊名, 出版年份, 卷号(期号): 起始-截止页

2.	专著  作者. 书名. 版本(第一版不写). 出版城市: 出版社, 出版年份: 起始-截止页

3.	论文集  论文作者. 论文题目//编者. 论文集名: 其他题名信息. 出版城市(或者会议城市): 出版者, 出版年: 引文起始-截止页码

4.	学位论文  作者. 学位论文题名. 城市: 论文保存单位, 年份

5.	网络文献  作者. 题名[文献类型标志/文献载体标志]. 出版地: 出版者, 出版年(更新日期)[引用日期]. 获取和访问路径

*注意: 作者姓前名后, 超过3名作者列前3名, 后加“, 等”; 英文姓名, 姓前名后, 姓首字母大写, 名缩写; 文献的项目要完整, 各项的顺序和标点要和格式要求一致; 未公开发表的论文、报告不列入正式文献, 如有必要可在正文当页下加注。英文文献格式同上。参考文献在正文中按出现顺序用[1], [2]......在右上角标注, 放在“参考文献”中时, 用[1], [2], ...顺序标注。



\addcontentsline{toc}{chapter}{参考文献}
\fancypagestyle{plain}
{
	\fancyhf{}
	\fancyhead[RE,RO]{参考文献}
	\fancyhead[LE,LO]{北京大学本科生毕业论文}
	\fancyfoot[CO,CE]{~\thepage~}
	\renewcommand{\headrulewidth}{0.7pt}
	\renewcommand{\footrulewidth}{0pt}
}
\fancyhf{}
\fancyhead[RE,RO]{参考文献}
\fancyhead[LE,LO]{北京大学本科生毕业论文}
\fancyfoot[CO,CE]{~\thepage~}
\renewcommand{\headrulewidth}{0.7pt}
\renewcommand{\footrulewidth}{0pt}
\bibliographystyle{plain}
\bibliography{ref}


\newpage
\chapter*{本科期间的主要工作和成果}
\normalsize
\linespread{1}
\addcontentsline{toc}{chapter}{本科期间的主要工作和成果}
\fancypagestyle{plain}
{
	\fancyhf{}
	\fancyhead[RE,RO]{本科期间的主要工作和成果}
	\fancyhead[LE,LO]{北京大学本科生毕业论文}
	\fancyfoot[CO,CE]{~\thepage~}
	\renewcommand{\headrulewidth}{0.7pt}
	\renewcommand{\footrulewidth}{0pt}
}
\fancyhf{}
\fancyhead[RE,RO]{本科期间的主要工作和成果}
\fancyhead[LE,LO]{北京大学本科生毕业论文}
\fancyfoot[CO,CE]{~\thepage~}
\renewcommand{\headrulewidth}{0.7pt}
\renewcommand{\footrulewidth}{0pt}




\newpage	
\chapter*{致谢}
\linespread{1.5}
\addcontentsline{toc}{chapter}{致谢}
\fancypagestyle{plain}
{
	\fancyhf{}
	\fancyhead[RE,RO]{致谢}
	\fancyhead[LE,LO]{北京大学本科生毕业论文}
	\fancyfoot[CO,CE]{~\thepage~}
	\renewcommand{\headrulewidth}{0.7pt}
	\renewcommand{\footrulewidth}{0pt}
}
\fancyhf{}
\fancyhead[RE,RO]{致谢}
\fancyhead[LE,LO]{北京大学本科生毕业论文}
\fancyfoot[CO,CE]{~\thepage~}
\renewcommand{\headrulewidth}{0.7pt}
\renewcommand{\footrulewidth}{0pt}


\end{document}