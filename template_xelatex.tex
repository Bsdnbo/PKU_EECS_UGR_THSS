\documentclass[utf8, a4paper, 12pt]{book}
\newcommand\clearemptydoublepage{\clearpage{\thispagestyle{empty}\cleardoublepage}}
\usepackage{pdfpages}



%字体设置
\usepackage[boldfont]{xeCJK}
\usepackage{zhnumber}
\usepackage{fontspec,xunicode,xltxtra}
\XeTeXlinebreaklocale "zh"
\setmainfont{TimesNewRomanPSMT}

%针对Mac OS的字体名称,其他系统请自行寻找宋体、黑体、楷体
\setCJKfamilyfont{songti}[BoldFont=STSongti-SC-Bold, Mapping={fullwidth-stop}]{STSongti-SC-Regular} 
\newcommand{\songti}{\CJKfamily{songti}}
\setCJKmainfont[BoldFont=STSongti-SC-Bold, Mapping={fullwidth-stop}]{STSongti-SC-Regular}

\setCJKfamilyfont{kaiti}[BoldFont=STKaitiSC-Bold]{STKaitiSC-Regular}
\newcommand{\kaiti}{\CJKfamily{kaiti}}

\setCJKfamilyfont{heiti}{STHeitiSC-Medium}
\newcommand{\heiti}{\CJKfamily{heiti}}

%中文字号与pt换算
%二号22pt,三号16pt,四号14pt,小四12pt,五号10.5pt
%\fontsize{12pt}{18pt}\selectfont 即为小四1.5倍行距



\usepackage{amsmath}
\usepackage{bm}
\usepackage{amsfonts}
\usepackage{enumerate}
\usepackage{fancyhdr}
\pagestyle{fancy}
\renewcommand{\chaptermark}[1]{\markboth{\thechapter\ \ #1}{}}


\usepackage{cite}
\newcommand{\upcite}[1]{\textsuperscript{\cite{#1}}} %引用在右上角



\usepackage{multirow,booktabs,makecell}
\usepackage{graphicx}
\usepackage{caption}
\DeclareCaptionFont{pkueecs}{\songti\fontsize{10.5pt}{15.75pt}}%五号,宋体/Time new roman
\captionsetup{font=pkueecs, labelsep=space}
\captionsetup[figure]{name=图}
\captionsetup[table]{name=表}
\renewcommand{\thetable}{\arabic{table}} %表格和图片编号不分章节,直接1,2,3 ...
\renewcommand{\thefigure}{\arabic{table}}



\setcounter{tocdepth}{3}
\setcounter{secnumdepth}{3}


\usepackage{titlesec}%自定义章节标题

%使目录中有三级标题,即subsubsection
\renewcommand\thechapter{第\zhnum{chapter}章}
\titleformat{\chapter}
{\center\fontsize{22pt}{33pt}\heiti}
{\thechapter}
{2em}
{}%第一章  绪论(二号、黑体)

\renewcommand\thesection{\arabic{section}} % 使得不显示章名,只显示节名
\titleformat{\section}
{\raggedright\fontsize{16pt}{24pt}\bfseries\songti}
{\thesection.\quad}
{0pt}
{}%1. 第一级(三号、宋体/Time new roman、加粗)

\titleformat{\subsection}
{\raggedright\bfseries\fontsize{14pt}{21pt}\songti}
{\thesubsection\quad}
{0pt}
{}%1.1 第二级(四号,宋体/Time new roman,加粗)

\titleformat{\subsubsection}
{\raggedright\bfseries\fontsize{12pt}{18pt}\songti}
{\thesubsubsection\quad}
{0pt}
{}%1.1.1 第三级(小四,宋体/Time new roman,加粗)




\usepackage[titles]{tocloft} %自定义目录
\renewcommand\cftsecaftersnum{.} %一级目录号后加点
\renewcommand\contentsname{全文目录}
\renewcommand\cftchapnumwidth{4em}
\renewcommand\cftchapfont{\heiti}





\title{}
\author{}
\date{}
\begin{document}
%\includepdf[pages={1-6}]{Cover&ReviewTable&Statement.pdf}




\chapter*{摘要}

\setcounter{page}{1}
\pagenumbering{Roman}
\songti\fontsize{12pt}{18pt}\selectfont %小四号,宋体/Time new roman,1.5倍行距

\bigskip
\noindent{\bfseries\songti 关键词: } 



\addcontentsline{toc}{chapter}{摘要} %手动加入目录
\fancypagestyle{plain} %因为latex默认每章第一页是plain所以需要重置一下plain和说明统一
{
	\fancyhf{} %清空
	
	\fancyhead[LE,RO]{摘要}
	%偶数页左页眉,奇数页右页眉均为“摘要”,及章名\leftmark

	\fancyhead[RE,LO]{北京大学本科生毕业论文}
	%偶数页右页眉,奇数页左页眉均为“北京大学本科生毕业论文”
	
	\fancyfoot[CO,CE]{~\thepage~}
	%偶数页和奇数页中页脚为页码,从对称考虑,因为每页在说明中都是一样的,不分奇偶
	
	\renewcommand{\headrulewidth}{0.7pt} %页眉线宽度,可调,不太清楚说明中是多少,待改
	
	\renewcommand{\footrulewidth}{0pt} %页脚线宽度为0,既没有
}

%默认的风格是fancy,设置于下,用于每章非第一页
\fancyhf{}
\fancyhead[RE,RO]{摘要}
\fancyhead[LE,LO]{北京大学本科生毕业论文}
\fancyfoot[CO,CE]{~\thepage~}
\renewcommand{\headrulewidth}{0.7pt}
\renewcommand{\footrulewidth}{0pt}
\clearemptydoublepage





\chapter*{\bfseries Abstract}
\fontsize{10.5pt}{15.75pt}\selectfont %5号,Time new roman,1.5倍行距
\bigskip
\noindent
{\bfseries Key Words: }


	
\addcontentsline{toc}{chapter}{\bfseries Abstract} %Abstract加粗
\fancypagestyle{plain}
{
	\fancyhf{}
	\fancyhead[LE,RO]{Abstract}
	\fancyhead[RE,LO]{北京大学本科生毕业论文}
	\fancyfoot[CO,CE]{~\thepage~}
	\renewcommand{\headrulewidth}{0.7pt}
	\renewcommand{\footrulewidth}{0pt}
}
\fancyhf{}
\fancyhead[LE,RO]{Abstract}
\fancyhead[RE,LO]{北京大学本科生毕业论文}
\fancyfoot[CO,CE]{~\thepage~}
\renewcommand{\headrulewidth}{0.7pt}
\renewcommand{\footrulewidth}{0pt}
\clearemptydoublepage






\fancypagestyle{plain}
{
	\fancyhf{}
	\fancyhead[LE,RO]{全文目录}
	\fancyhead[RE,LO]{北京大学本科生毕业论文}
	\fancyfoot[CO,CE]{~\thepage~}
	\renewcommand{\headrulewidth}{0.7pt}
	\renewcommand{\footrulewidth}{0pt}
}
\fancyhf{}
\fancyhead[LE,RO]{全文目录}
\fancyhead[RE,LO]{北京大学本科生毕业论文}
\fancyfoot[CO,CE]{~\thepage~}
\renewcommand{\headrulewidth}{0.7pt}
\renewcommand{\footrulewidth}{0pt}
\renewcommand{\contentsname}{\centerline{全文目录}}
\renewcommand{\listfigurename}{\centerline{图目录}}
\renewcommand{\listtablename}{\centerline{表目录}}
\tableofcontents
\addcontentsline{toc}{chapter}{全文目录}
\clearemptydoublepage

\listoftables
\clearemptydoublepage

\listoffigures
\clearemptydoublepage






\fancypagestyle{plain}
{
	\fancyhf{}
	\fancyhead[RO]{\leftmark}
	\fancyhead[LE]{\rightmark}
	\fancyhead[RE,LO]{北京大学本科生毕业论文}
	\fancyfoot[RO,LE]{~\thepage~}
	\renewcommand{\headrulewidth}{0.7pt}
	\renewcommand{\footrulewidth}{0pt}
}
\fancyhf{}
\fancyhead[RO]{\leftmark}
\fancyhead[LE]{\rightmark}
\fancyhead[RE,LO]{北京大学本科生毕业论文}
\fancyfoot[RO,LE]{~\thepage~}
\renewcommand{\headrulewidth}{0.7pt}
\renewcommand{\footrulewidth}{0pt}




\pagenumbering{arabic}
\setcounter{page}{1}
\fontsize{12pt}{18pt}\selectfont



\chapter{章节名称}
\section{一级段落名称}
\subsection{二级段落名称}
\subsubsection{三级段落名称}
\fontsize{12pt}{18pt}\selectfont
引用文献如\upcite{PhysRev.47.777},引用表如表\ref{tab:input_output_r},引用图如图\ref{fig:sample}。



\begin{table}[h]
\songti\fontsize{10.5pt}{15.75pt}\selectfont
\centering
\caption{不同频率下的输入和输出阻抗}
\label{tab:input_output_r}
\begin{tabular}{cccc} %表格使用三线表
\toprule %不确定说明中三条线的粗细,待改
频率(Hz) & 1 & 10k & 1M \\
\midrule
输入电阻($\Omega/^\circ$) & 339.719k/-87.84 & 5.6707k/-9.827 & 351.188/-72.377\\
输出电阻($\Omega/^\circ$) & 338.638k/-89.663 & 1.9866k/-1.1228 & 1.9189k/-14.801 \\
\bottomrule
\end{tabular}
\songti\fontsize{12pt}{18pt}\selectfont
\end{table}

\begin{figure}[h]
\songti\fontsize{10.5pt}{15.75pt}\selectfont
\centering
\includegraphics[width=12cm]{sample.jpg}
\caption{示例图片}
\label{fig:sample}
\songti\fontsize{12pt}{18pt}\selectfont
\end{figure}
\clearemptydoublepage




\chapter*{参考文献}
\begin{enumerate}[{[1]}]
\songti\fontsize{10.5pt}{10.5pt}\selectfont %正文,五号,中文宋体,英文Time new roman,1倍行距
	\item 期刊
	\item 网络文档
\end{enumerate}

1.	期刊  作者. 论文名. 刊名, 出版年份, 卷号(期号): 起始-截止页

2.	专著  作者. 书名. 版本(第一版不写). 出版城市: 出版社, 出版年份: 起始-截止页

3.	论文集  论文作者. 论文题目//编者. 论文集名: 其他题名信息. 出版城市(或者会议城市): 出版者, 出版年: 引文起始-截止页码

4.	学位论文  作者. 学位论文题名. 城市: 论文保存单位, 年份

5.	网络文献  作者. 题名[文献类型标志/文献载体标志]. 出版地: 出版者, 出版年(更新日期)[引用日期]. 获取和访问路径

*注意: 作者姓前名后, 超过3名作者列前3名, 后加“, 等”; 英文姓名, 姓前名后, 姓首字母大写, 名缩写; 文献的项目要完整, 各项的顺序和标点要和格式要求一致; 未公开发表的论文、报告不列入正式文献, 如有必要可在正文当页下加注。英文文献格式同上。参考文献在正文中按出现顺序用[1], [2]......在右上角标注, 放在“参考文献”中时, 用[1], [2], ...顺序标注。



\addcontentsline{toc}{chapter}{参考文献}
\fancypagestyle{plain}
{
	\fancyhf{}
	\fancyhead[LE,RO]{参考文献}
	\fancyhead[RE,LO]{北京大学本科生毕业论文}
	\fancyfoot[RO,LE]{~\thepage~}
	\renewcommand{\headrulewidth}{0.7pt}
	\renewcommand{\footrulewidth}{0pt}
}
\fancyhf{}
\fancyhead[LE,RO]{参考文献}
\fancyhead[RE,LO]{北京大学本科生毕业论文}
\fancyfoot[RO,LE]{~\thepage~}
\renewcommand{\headrulewidth}{0.7pt}
\renewcommand{\footrulewidth}{0pt}
\clearemptydoublepage





\bibliographystyle{plain}
\bibliography{ref}
这是参考“参考文献”,主要用来看引用的顺序,请手动些参考文献或自行写程序,最终编译请删除
\clearemptydoublepage




\chapter*{本科期间的主要工作和成果}
\songti\fontsize{12pt}{12pt}\selectfont %小四号,中文宋体,英文Time new roman,1倍行距

本科期间参加的主要科研项目

\noindent 本研基金
\begin{enumerate}
	\item 基金名称. 基金类型. 指导老师. 基金支持年限
\end{enumerate}

\noindent 各种科研项目
\begin{enumerate}
	\item 项目名称. 项目类型
\end{enumerate}

格式下

期刊:

全部作者. 论文名. 期刊名, 出版年份, 卷号(期号): 起始-截止页

会议论文:

全部作者. 论文名. 会议名, 会议举办地, 会议举办时间, 起始-截止页

专利

全部专利申请人. 专利名称. 专利申请号. 专利申请日期. 国别



\addcontentsline{toc}{chapter}{本科期间的主要工作和成果}
\fancypagestyle{plain}
{
	\fancyhf{}
	\fancyhead[LE,RO]{本科期间的主要工作和成果}
	\fancyhead[RE,LO]{北京大学本科生毕业论文}
	\fancyfoot[RO,LE]{~\thepage~}
	\renewcommand{\headrulewidth}{0.7pt}
	\renewcommand{\footrulewidth}{0pt}
}
\fancyhf{}
\fancyhead[LE,RO]{本科期间的主要工作和成果}
\fancyhead[RE,LO]{北京大学本科生毕业论文}
\fancyfoot[RO,LE]{~\thepage~}
\renewcommand{\headrulewidth}{0.7pt}
\renewcommand{\footrulewidth}{0pt}
\clearemptydoublepage




\chapter*{致谢}
\songti\fontsize{12pt}{18pt}\selectfont %正文,小四号,中文宋体,英文Time new roman,1.5倍行距


\addcontentsline{toc}{chapter}{致谢}
\fancypagestyle{plain}
{
	\fancyhf{}
	\fancyhead[LE,RO]{致谢}
	\fancyhead[RE,LO]{北京大学本科生毕业论文}
	\fancyfoot[RO,LE]{~\thepage~}
	\renewcommand{\headrulewidth}{0.7pt}
	\renewcommand{\footrulewidth}{0pt}
}
\fancyhf{}
\fancyhead[LE,RO]{致谢}
\fancyhead[RE,LO]{北京大学本科生毕业论文}
\fancyfoot[RO,LE]{~\thepage~}
\renewcommand{\headrulewidth}{0.7pt}
\renewcommand{\footrulewidth}{0pt}
\clearemptydoublepage




\end{document}