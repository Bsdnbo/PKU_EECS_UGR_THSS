\small
\linespread{1}\selectfont
%正文,五号,中文宋体,英文Time new roman,1倍行距
\chapter*{参考文献}
\noindent
\begin{enumerate}[{[1]}]
\small
	\item 期刊
	\item 网络文档
\end{enumerate}

1.	期刊  作者. 论文名. 刊名, 出版年份, 卷号(期号): 起始-截止页

2.	专著  作者. 书名. 版本(第一版不写). 出版城市: 出版社, 出版年份: 起始-截止页

3.	论文集  论文作者. 论文题目//编者. 论文集名: 其他题名信息. 出版城市(或者会议城市): 出版者, 出版年: 引文起始-截止页码

4.	学位论文  作者. 学位论文题名. 城市: 论文保存单位, 年份

5.	网络文献  作者. 题名[文献类型标志/文献载体标志]. 出版地: 出版者, 出版年(更新日期)[引用日期]. 获取和访问路径

*注意: 作者姓前名后, 超过3名作者列前3名, 后加“, 等”; 英文姓名, 姓前名后, 姓首字母大写, 名缩写; 文献的项目要完整, 各项的顺序和标点要和格式要求一致; 未公开发表的论文、报告不列入正式文献, 如有必要可在正文当页下加注。英文文献格式同上。参考文献在正文中按出现顺序用[1], [2]......在右上角标注, 放在“参考文献”中时, 用[1], [2], ...顺序标注。



\addcontentsline{toc}{chapter}{参考文献}
\fancypagestyle{plain}
{
	\fancyhf{}
	\fancyhead[RE,RO]{参考文献}
	\fancyhead[LE,LO]{北京大学本科生毕业论文}
	\fancyfoot[CO,CE]{~\thepage~}
	\renewcommand{\headrulewidth}{0.7pt}
	\renewcommand{\footrulewidth}{0pt}
}
\fancyhf{}
\fancyhead[RE,RO]{参考文献}
\fancyhead[LE,LO]{北京大学本科生毕业论文}
\fancyfoot[CO,CE]{~\thepage~}
\renewcommand{\headrulewidth}{0.7pt}
\renewcommand{\footrulewidth}{0pt}


\bibliographystyle{unsrt}
\bibliography{ref}
这是参考“参考文献”,主要用来看引用的顺序,请手动些参考文献或自行写程序,最终编译请删除
\clearpage